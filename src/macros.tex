\newcommand{\sys}{{NetShaper}}
\newcommand{\nsc}{network side channel}
\newcommand{\nsca}{network side-channel}
\newcommand{\eg}{e.g.,~}
\newcommand{\ie}{i.e.,~}
\newcommand{\medvid}{MedFlix}
\newcommand{\unshapedQ}{Q}
\newcommand{\flowmap}{\texttt{FlowMap}}
\newcommand{\prepare}{\texttt{Prepare}}
\newcommand{\ushaper}{\texttt{UShaper}}
\newcommand{\dshaper}{\texttt{DShaper}}

\newcommand{\istream}{S}
%\newcommand{\ostream}{S^{\epsilon,\delta}}
\newcommand{\ostream}{O}
\newcommand{\istreamw}{S_w}
\newcommand{\ostreamw}{O_w}
\newcommand\norm[1]{\|#1\|}
\newcommand{\ssens}{\Delta_{\winlen}}
\newcommand{\qsens}{\Delta_\dpintvl}
\newcommand{\winlen}{W}
\newcommand{\window}{w}
\newcommand{\stream}{S}
\newcommand{\streamw}[1]{S_{#1}}
\newcommand{\qlen}{L}
\newcommand{\dpintvl}{T}
\newcommand{\qlent}[1]{L_{#1}}
\newcommand{\qlendp}{\tilde{L}}
\newcommand{\qlendpt}[1]{\tilde{L}_{#1}}
%\newcommand{\qlendp}{L^{\epsilon_\dpintvl, \delta_\dpintvl}}
%\newcommand{\qlendpt}[1]{L_{#1}^{\epsilon_\dpintvl, \delta_\dpintvl}}
\newcommand{\payload}{R}
\newcommand{\dummy}{D}
\newcommand{\numupdates}{\lceil \frac{\winlen}{\dpintvl} \rceil}
\newcommand{\varnumupdates}{N}

\newcommand{\base}{{\bf Base}}
\newcommand{\nsnoshape}{{\bf NS\textsubscript{M}}}
\newcommand{\ns}{{\bf NS}}
\newcommand{\nssim}{{\bf NS\textsubscript{S}}}
\newcommand{\constshape}{{\bf CR}}
\newcommand{\pacer}{{\bf Pacer}}

\newcommand{\parasum}[1]{\smallskip\noindent\hl{#1}}

\newcommand{\todo}[1]{\textcolor{red} {#1}}
\newcommand{\update}[1]{\textcolor{blue} {#1}}
\newcommand{\final}[1]{\textcolor{blue} {#1}}
\newcommand{\citeme}[1]{\textcolor{red} {[{\bf #1}]}}
\newcommand{\am}[1]{\textcolor{cyan}{\bf AM: #1}}
\newcommand{\mis}[1]{\textcolor{purple}{\bf MIS: #1}}
\newcommand{\ml}[1]{\textcolor{orange}{\bf ML: #1}}
\newcommand{\as}[1]{\textcolor{forestgreen}{\bf AS: #1}}
\newcommand{\strike}[1]{\sout{#1}}
\newcommand{\addref}{\textcolor{orange}{\textbf{ADD REF}}}

% \newcommand{\todo}[1]{{#1}}
% \newcommand{\update}[1]{{#1}}
% \newcommand{\final}[1]{{#1}}
% \newcommand{\citeme}[1]{\cite{#1}}
% \newcommand{\am}[1]{#1}
% \newcommand{\mis}[1]{{#1}}
% \newcommand{\ml}[1]{{#1}}
% \newcommand{\as}[1]{{#1}}
% %\newcommand{\strike}[1]{}




\newcommand{\NA}{\textsc{n/a}}	% for "not applicable"
\newcommand{\etal}{\emph{et al}}

% Some useful macros for typesetting terms.
\newcommand{\file}[1]{\texttt{#1}}
\newcommand{\class}[1]{\texttt{#1}}
\newcommand{\latexpackage}[1]{\href{http://www.ctan.org/macros/latex/contrib/#1}{\texttt{#1}}}
\newcommand{\latexmiscpackage}[1]{\href{http://www.ctan.org/macros/latex/contrib/misc/#1.sty}{\texttt{#1}}}
\newcommand{\env}[1]{\texttt{#1}}
\newcommand{\BibTeX}{Bib\TeX}

% Define a command \doi{} to typeset a digital object identifier (DOI).
% Note: if the following definition raise an error, then you likely
% have an ancient version of url.sty.  Either find a more recent version
% (3.1 or later work fine) and simply copy it into this directory,  or
% comment out the following two lines and uncomment the third.
\DeclareUrlCommand\DOI{}
\newcommand{\doi}[1]{\href{http://dx.doi.org/#1}{\DOI{doi:#1}}}
%\newcommand{\doi}[1]{\href{http://dx.doi.org/#1}{doi:#1}}

% Useful macro to reference an online document with a hyperlink
% as well with the URL explicitly listed in a footnote
% #1: the URL
% #2: the anchoring text
\newcommand{\webref}[2]{\href{#1}{#2}\footnote{\url{#1}}}

% epigraph is a nice environment for typesetting quotations
\makeatletter
\newenvironment{epigraph}{%
	\begin{flushright}
	\begin{minipage}{\columnwidth-0.75in}
	\begin{flushright}
	\@ifundefined{singlespacing}{}{\singlespacing}%
    }{
	\end{flushright}
	\end{minipage}
	\end{flushright}}
\makeatother

% \FIXME{} is a useful macro for noting things needing to be changed.
% The following definition will also output a warning to the console
\newcommand{\FIXME}[1]{\typeout{**FIXME** #1}\textbf{[FIXME: #1]}}

% END
