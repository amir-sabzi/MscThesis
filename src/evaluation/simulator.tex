The simulator has 4 major parts: 
Inbound traffic processor where it receives and processes the collected traces.
Streaming application, which emulates the behavior of the application that generated traffic traces.
Single-Producer Single-Consumer Queue represents the concept of \textit{buffering queue} that we introduce in \Cref{sec:dp-shaping-definitions} as part of our shaping mechanism.
Finally, Traffic Shaping Module implements the core shaping mechanism we explained in \Cref{subsec:dp-shaping-mechanism}.


\paragraph{Inbound Traffic Processor}\label{subsubsec:design-sim-inbound}
The simulator takes a PCAP file as the input. 
A PCAP (Packet Capture) file is a data file format used to store network packet data captured during network traffic analysis and packet sniffing.
These files contain a record of various network packets exchanged between devices on a network.
Particularly, a PCAP file contains the size and timestamp for every single packet that was part of the network traffic.
To reconstruct the traffic shape from PCAP file, the simulator discretizes time based on a user defined parameter, time resolution.
Time resolution defines the finest level of time granularity upon which simulator operators rely.
For instance, if we define this resolution as $1ms$, simulator maps the input trace to a time series, where $i$\textsuperscript{th} element of time series represent the number of bytes transmitted between $i$\textsuperscript{th} and $(i+1)$\textsuperscript{th} millisecond of data transmission.
As the data time resolution decreases, the simulator more closely replicates the actual shape of the input traffic traces.
We implement this functionality, as well as preprocessing of the input data, as part of the \texttt{DataProcessor} class in simulator code~\cite{netshaper_repo}.



\paragraph{Streaming Application}\label{subsubsec:design-simulator-app}
We develop a simulated application that transmits data based on the time series extracted from PCAP files. 
More precisely, at intervals defined by time resolution of the simulator, the application transmits a quantity of bytes determined by the input time series.
The application continues sending bytes until it either finishes input traces or it receives a termination signal from traffic shaping module.
Based on the input trace, simulated application can simulate both server and client side traffic shape.
We implement the simulated application as \texttt{Application} class in simulator.

\paragraph{Single-Producer Single-Consumer Queues}\label{subsubsec:design-sim-spsc-queue}
In \Cref{sec:dp-shaping-definitions}, we introduce the concept of \textit{buffering queue} to control the maximum information accessible by adversary within a predefined window of time.
We implement this concept in simulator as a simple single-consumer single-producer queue.
The producer is the streaming application we explained in \Cref{subsubsec:design-simulator-app}.
The consumer of this queue is our DP traffic shaping module.
The simulated application, as the producer for this queue, pushes specified number of bytes based on the input trace into the queue.
We define a parameter, \texttt{shaping interval}, corresponding to the intervals of $T$ seconds we introduced in \Cref{subsec:dp-shaping-mechanism}.
Every $T$ seconds, the shaping module dequeues data with a size determined by
DP traffic shaping mechanism.



\paragraph{Traffic Shaping Module}
Traffic shaping module is the core part of our simulator where we implement the differentially private traffic shaping mechanism.
Every $T$ seconds, traffic shaping module determines the number of bytes to be transmitted as the shaped traffic based on a DP measurement on the size of the application queue, $\qlendp$.
We define DP parameters and frequency of measurement in the simulator configuration file. 
Then, it dequeues $\payload = \min(\qlendp, \qlen)$ from the queue, where $\qlen$ is the number of available bytes in the queue at the measurement time. 
If $\qlendp$ is larger that $\payload$, the traffic shaping module adds $\dummy = \qlendp - \payload$ bytes of dummy to the transmitted traffic.
We, further, extend the traffic shaping module to implement constant-rate and Pacer~\cite{mehta2022pacer} traffic shaping to compare overheads and privacy of these approaches with {\sys}. 

