We design and implement a traffic shaping simulator to evaluate the bandwidth overhead and privacy guarantees of {\sys}'s.
The simulator helps users to find parameters of {\sys} shaping mechanism that balance the trade-offs between privacy and overhead according to their requirement.
They can use these parameters in the actual {\sys} system, which we designed in \Cref{sec:tunnel-design} and implemented as part of NetShaper paper \cite{netshaper}.
The simulator specifically simulates the functionality of \emph{Outbound Traffic Shaping} component of a tunnel endpoint to evaluate the privacy and bandwidth overheads of our DP shaping strategy.
The simulator consists of four major components: the Inbound Traffic Processor, responsible for receiving and processing collected traces; the Streaming Application, which emulates the behavior of the application that generated the traffic traces; the Single-Producer Single-Consumer Queue, representing the concept of the buffering queue introduced in \Cref{sec:dp-shaping-definitions} as part of our shaping mechanism; and finally, the Traffic Shaping Module, which implements the core shaping mechanism explained in \Cref{subsec:dp-shaping-mechanism}.


\paragraph{Inbound Traffic Processor}\label{subsubsec:design-sim-inbound}
The simulator takes a PCAP file as the input. 
A PCAP file contains the size and timestamp for every packet that exchanged during a communication session.
To reconstruct the traffic shape from a PCAP file, the simulator discretizes time based on a user defined parameter, \emph{time resolution}.
The \emph{Time resolution} defines the finest level of time granularity upon which the simulator relies.
For instance, if we define \emph{time resolution} as $1ms$, the simulator maps the input trace to a time series, where the $i$\textsuperscript{th} element of time series represent the number of bytes transmitted between the $i$\textsuperscript{th} and the $(i+1)$\textsuperscript{th} millisecond of data transmission (the time series is zero when nothing is sent).
As the data \emph{time resolution} decreases, the simulator more closely replicates the actual shape of the input traffic.
We implement this functionality, as well as preprocessing of the input data, as part of the \texttt{DataProcessor} class in simulator~\cite{netshaper_repo}.



\paragraph{Streaming Application}\label{subsubsec:design-simulator-app}
We develop a simulated application that transmits data based on the time series extracted from PCAP files. 
More precisely, at intervals defined by the \emph{time resolution} of the simulator, the application transmits a quantity of bytes determined by the input time series.
The application continues sending bytes until it either finishes the input traces or it receives a termination signal from the traffic shaping module.
Based on the input trace, simulated application can simulate both server and client side traffic shape.
We implement the simulated application as the \texttt{Application} class in the simulator.

\paragraph{Single-Producer Single-Consumer Queues}\label{subsubsec:design-sim-spsc-queue}
In \Cref{sec:dp-shaping-definitions}, we introduced the \textit{buffering queue} to control the maximum information accessible by an adversary in a predefined time window.
We implement this in the simulator as a simple single-consumer, single-producer queue.
The producer is the streaming application, and
the consumer is our DP traffic shaping module.
The simulated application, as the producer for this queue, pushes a specified number of bytes, based on the input trace, into the queue.
We define a parameter, \emph{shaping interval}, corresponding to the intervals of $T$ seconds we introduced in \Cref{subsec:dp-shaping-mechanism}.
Every $T$ seconds, the shaping module dequeues data with a size determined by
the DP traffic shaping mechanism.



\paragraph{Traffic Shaping Module}
The traffic shaping module is the core part of our simulator where we implement the differentially private traffic shaping mechanism.
Every $T$ seconds, the traffic shaping module determines the number of bytes to be transmitted based on a DP measurement on the size of the application queue, $\qlendp$.
Then, it dequeues $\payload = \min(\qlendp, \qlen)$ from the queue, where $\qlen$ is the number of available bytes in the queue at the measurement time. 
If $\qlendp$ is larger that $\payload$, the traffic shaping module adds $\dummy = \qlendp - \payload$ bytes of dummy data to the transmitted traffic.
We also extend the traffic shaping module to implement constant-rate and Pacer~\cite{mehta2022pacer} traffic shaping to allow comparison between these approaches and {\sys}. 

