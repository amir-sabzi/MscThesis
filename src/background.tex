\chapter{Background and Prior Work}\label{ch:background}
% Outline for the section:
% - Network Side Channels
% 	- Overview
% 	- Target application: Video Streaming
% 	- CNN classifier (BB)

% - Defenses
% 	- Path splitting approaches
% 	- Adversarial ML approaches
% 	- Shaping based approaches
% 		- Static shaping
% 		- Dynamic shaping

% - Differential Privacy Overview

% - QUIC Overview (I am not convinced yet this is necessary.)

% - Threat Model
% Background intro. Explaining what you've included in this section. 
% TODO: update if you add more to background

We begin this chapter by discussing the problem of network side-channel attacks in \Cref{sec:ns-attacks} and exploring various proposed mitigations for such attacks in \Cref{sec:ns-defenses}.
We, then, present a comprehensive definition of differential privacy (DP), outline its key properties, and highlight its primary applications, thereby setting the foundation for proposing our differentially private traffic shaping mechanism in \Cref{sec:dp-background}.
Finally, in \Cref{sec:background-quic} we conclude this chapter by providing a brief overview of QUIC protocol, which plays a pivotal role in the design of {\sys}.

\section{Network Side Channel Attacks}\label{sec:ns-attacks}
% What is a side channel?
In this section, we provide an overview of network side-channel attacks. 
We, further, discuss video streaming as one specific target application of side-channel attacks, and explain the state-of-the-art traffic analysis attack proposed for this application and elaborate on its strengths and weaknesses. 

A side channel is a method of extracting information from a program by observing its functionality and environment, rather than through its intended input or output. 
Extracting information from side channels becomes particularly viable when a program has shared resources with other untrusted entities. 
When an adversary shares resources with the victim's program, it can observe victim's resource usage pattern.
The adversary then can leverage the correlation between resource usage pattern (side channels) and victims' secrets to breach their privacy.

% Essential Steps in side channel attacks
A side-channel attack comprises two essential steps: Profiling and Inference.
During profiling, the adversary extracts the correlation between the victim's secrets and their resource usage patterns.
This involves monitoring various resource usage patterns that emerge during the program's execution.
In the inference step, the adversary utilizes the prior knowledge acquired in the first step to infer the victim's secrets based on the observations of resource usage patterns.
One trivial solution to address side-channel attack is to isolate the program and all its dedicated resources such as CPUs, memory, storage, and network resources at all possible level down to hardware. 
However, this approach often results in inefficient resource utilization, as the resources that the program does not currently use remain idle. 
Moreover, many programs rely on inherently shared resources, such as a common network infrastructure.
Therefore, mitigating side-channels attacks is intrinsically challenging. 

% What are network side channels 
Network applications, such as web browsers, email clients, video conferencing software, file-sharing services, and online gaming platforms, are exceedingly popular these days.
These applications consist of a service that communicates with clients over an encrypted communication on the network.
However, the encryption does not conceal packet sizes and timing transmitted by an application, which is correlated with users' sensitive information in many applications.
In a network side-channel attack, by utilizing this correlation, an adversary with control over the underlying network links (e.g., Internet Service Providers) can monitor the traffic pattern and potentially reveal the content of the communication.
Recent advancements in machine learning have significantly empowered the inference step of network side-channel attacks, providing adversaries with enhanced capabilities to effectively map observations to the victim's sensitive information~\cite{schuster2017beautyburst, bhat2019varcnn, hayes2016kfp, sirinam2018df}.


% \subsection{Website Fingerprinting}\label{subsec:web-fingerprinting}

\subsection{Video identification}\label{subsec:video-classification}
Recently, video services have gained immense popularity, becoming an integral part of people's daily Internet activities.
Video-sharing platforms, such as YouTube, and commercial online streaming platforms, such as Netflix, constitute a significant portion of the world's total network traffic.
In 2023, video services account for 82.5\% of all web traffic, making them by far the most popular type of content over the internet~\cite{webstat}.
The popularity of these services makes them a natural target for traffic analysis attacks.
Video streams are characterized by bursty traffic patterns, which often involves short-lived periods of high rate data transmission interspersed with relatively longer periods of low or moderate rate network activity.
When these patterns are correlated with specific content, an adversary capable of measuring them may have the ability to identify the exact video being streamed.
In this section, we first provide an overview of the widely used video streaming protocol, Dynamic Adaptive Streaming over HTTP (MPEG-DASH), and examine how it can potentially result in information leakage. 
Next, we elaborate on the state-of-the-art network side-channel attack, known as Beauty and the Burst~\cite{schuster2017beautyburst}, which leverages bursty pattern of MPEG-DASH protocol to reveal users' information. 

In MPEG-DASH protocol, the video server encodes the video content and divide it into short segments, ranging from few seconds to few tens of seconds.
Then, it creates a Manifest file to store information about available data segments, their quality, and a URL to access them.
Upon receiving a request for a video from a client, the video server sends the Manifest file to the client. 
Following this, the client sends requests based on the URLs provided in the Manifest file to download segments corresponding to the desired quality.
The segment sizes and the timing of each segment can collectively create a unique pattern for a video.
An adversary with control over network link can measure segment sizes and their temporal pattern to identify the streaming video.
In Beauty and the Burst paper~\cite{schuster2017beautyburst}, the authors show that even a malicious extension in a browser, can extract segment sizes and timing of a video that the user is watching.
Based on the MPEG-DASH standard, Beauty and Burst attack model traffic traces as time-series.
Specifically, each traffic segment can be represented as a tuple $(t_i, b_i)$, where $t_i$ denotes the segment's transmission time and $b_i$ represents its size. 
A traffic analysis attack involves the attempt to infer the content of traffic from its observed pattern, which is essentially a sequence modeling task.
At the inference stage, Beauty and the burst~\cite{schuster2017beautyburst} utilizes a convolutional neural network (CNN) architecture to classify videos based on the time-series representation of observed traffic.
The architecture of Burst and Beauty model is represented in \Cref{fig:bandb-arch}. 
We evaluate this architecture with our dataset in section \Cref{sec:eval-empirical-privacy}.
\begin{figure}[t]
  \centering
  %\includegraphics[width=\columnwidth]{figures/DPshaping_concept_vertical.pdf}
  %\includegraphics[width=\columnwidth]{figures/DPshaping_concept_horizontal.pdf}
  \includegraphics[width=\columnwidth]{figures/BandB_arch.pdf}
  \caption{Beauty and The Burst CNN model architecture.}
  \label{fig:bandb-arch}
\end{figure}

\subsection{Website Fingerprinting}
Website fingerprinting is similar to video identification in essence.
In website fingerprinting attacks, the attacker goal is to map the victim's packet sequence to the website victim visits at the time. 
Deep Neural Networks (DNNs) are shown to be effective in website fingerprinting attacks~\cite{sirinam2018df}.
However, due to the differences in characteristics of website traffic as compared to video streaming applications, website fingerprinting attacks use different set of features and model architectures. 
A notable contrast distinguishing video from web traffic traces is the comparatively shorter duration of web traces as opposed to the more extended durations typical of video traces.
Sirinam~\etalc{sirinam2018df} show that a small CNN with one convolutional layer is able to map users' traffic to their content with 98\% accuracy.
They further show that their model is able to achieve 90\% accuracy in presence of WTF-PAD~\cite{juarez2016wtfpad} defense mechanism, and it achieves 49.7\% accuracy when Walke-Talkie is deployed.
This highlights the ineffectiveness of ad-hoc shaping mechanism in face of emerging attacks.



\section{Network Side Channel Defenses}\label{sec:ns-defenses}
% Section Intro, elaborate on the general theme of the section. 
Now, we explore mitigation approaches proposed to address network side-channel attacks. 
We categorize these approaches into three general classes: (i) Traffic shaping with static profiling (\Cref{subsec:static-traffic-shaping}), (iii) Traffic shaping without pre-profiling (\Cref{subsec:dynamic-traffic-shaping}), (iii) Adversarial Noise (\Cref{subsec:adversarial-noise}).
Our method, the DP traffic shaping, falls into the second category. 
We compare {\sys}'s shaping mechanism with these approaches and highlight their respective strengths and weaknesses.



\section{Traffic Shaping with Static Profiling}\label{subsec:static-traffic-shaping}
These methods typically consist of two distinct phases: the profiling phase and the shaping phase.
The profiling phase is often performed before data transmission on collected network traces.  
During the profiling phase, the shaping mechanism collects user-specific information from the user's traffic traces.
This information can vary and may include details such as the type of traffic pattern (\ie bursty or continuous), distribution of packet sizes, or timing of packet transmission.
The shaping mechanism then leverages this information during the shaping phase to transform users' traffic into a new pattern that ensures privacy with minimal overhead.
We discuss three methods that require static profiling: Traffic Morphing~\cite{wright2009traffic}, Walkie-Talkie~\cite{wang2017walkie}, and Pacer~\cite{mehta2022pacer}.
We chose these three techniques as they represent different approaches to static traffic shaping.
We start with Traffic Morphing in \Cref{subsubsec:traffic-morphing}.
Traffic Morphing changes the distribution of packet sizes of a webpage in a way that they resemble packets generated by another webpage.
In \Cref{subsubsec:walkie-talkie}, we discuss Walkie-Talkie technique.
Walkie-Talkie represents clustering methods, which is one of the most common approaches for traffic shaping with static profiling.
Finally, Pacer is the state-of-the-art in this category of traffic shaping mechanisms, which offers rigorous privacy guarantees while shaping traffic with small overhead.
We provide an overview of Pacer traffic shaping technique in \Cref{subsubsec:background-defenses-pacer}.   

% \todo{Move the next sentence to the end of related work}
% However, scaling static traffic shaping methods is challenging, or even impossible in many cases, due to the need for profiling numerous traffic traces.
% methods we are going to cover:
%% Walkie-Talkie 
%% Supersequence
%% Glove
%% Traffic morphing 
\subsection{Traffic Morphing}\label{subsubsec:traffic-morphing}
In traffic morphing~\cite{wright2009traffic}, the shaping mechanism alters the distribution of packet sizes in such a way that they resemble packets generated by a different website rather than the original website.
Consequently, the attacker fails to map the traffic shape to its content. 
Traffic Morphing (TM) operates based on a technique known as \textit{Direct Target Sampling}.

First it collects the distribution of packets sizes for all websites in the dataset.
Then, Given a webpage $W$, TM randomly selects another webpage $W'$ from the dataset.
When selecting the target webpage $W'$, TM uses an optimization method to choose webpages that result in minimal overhead. 
Upon receiving each packet produced by $W$ with size $l$, TM samples a packet size $l'$ from the distribution of $W'$. 
If $l' > l$, TM pads the outgoing packet with dummy data to match the size $l'$. 
On the other hand, if $l' < l$, TM sends $l'$ bytes and continues sampling from the distribution of $W'$ until all bytes of the original packet have been transmitted.
It is important to emphasize that, unlike {\sys}, the traffic morphing method does not hide the timing or duration of bursts of traffic.
Therefore, an adversary can still exploit these features of traffic traces to potentially reveal users' information, even when traffic morphing is deployed~\cite{dyer2012peek}.


\subsection{Walkie-Talkie}\label{subsubsec:walkie-talkie}
Wang\etalc{wang2017walkie} propose {Walkie-Talkie} as a new defense mechanism implemented in browsers to hide the traffic shape of sensitive websites.
{Walkie-Talkie} requires browser modification to change the default full-duplex communication to half-duplex mode.
In full-duplex communication, multiple servers are actively transmitting web page data to the client, while the client concurrently submits additional resource requests, potentially to different servers.
In half duplex mode, the client sends requests only after the web servers have satisfied all previous requests.
Using half-duplex mode enables {Walkie-Talkie} to change communication to a sequence of bursts of data between one client and one server at a time.
Walkie-Talkie uses a fixed interval to define a burst sequence as a time series, where the size of elements in the time series represents the number of bytes transmitted within the corresponding interval.
For any sensitive website $W$, {Walkie-Talkie} first extracts its burst sequence as a time-series, where each element in the time-series represents the size of the burst.  
Then, using an approach similar to Traffic Morphing~\cite{wright2009traffic}, {Walkie-Talkie} chooses a decoy website $W'$ with a potentially different burst sequence.
For every pair of a sensitive and decoy websites, Walkie-Talkie reshapes the $i$\textsuperscript{th} burst size of them to the maximum of the $i$\textsuperscript{th} burst size in the decoy and the sensitive website. 
Therefore, if an adversary observes the shaped burst sequences, it can not ascertain whether the users accessed $W$ or $W'$.  
Choosing multiple decoy burst sequences for any sensitive webpage can further increase the privacy of {Walkie-Talkie}.

Essentially, {Walkie-Talkie} uses a clustering technique for shaping traffic, conceptually similar to Supersequence~\cite{wang2014supersequence}, Glove~\cite{nithyanand2014glove}, and Tamaraw~\cite{cai2014systematic}.
It maps the traffic shapes of various web pages to the same pattern, effectively rendering them indistinguishable for potential adversaries.
The advantages of the {Walkie-Talkie} are: it adds small overhead compared to other defenses, requires minimal computation to extract the burst sequence of shaped traffic, and stores small metadata for decoy webpages.
It also has several disadvantages. 
First, {Walkie-Talkie} requires browser modification.  
Secondly, similar to most of the traffic shaping methods with static profiling, {Walkie-Talkie} identifies the burst sequence of a webpage before its transmission. 
This is particularly problematic when the burst sequence of a webpage changes due to the network conditions, such as congestion, or application flow control.  
Finally, the adoption of half-duplex mode in browsers imposes a notable constraint on the scalability of this approach, as numerous applications, such as video streaming and file downloading, require multiple and simultaneous communications with web servers.


\subsection{Pacer}\label{subsubsec:background-defenses-pacer}
When discussing our threat model for network side-channel attacks in \Cref{sec:threat-model}, we assumed that the adversary can monitor the underlying network links, thereby enabling it to observe the victim's traffic patterns.
However, this is not the most realistic adversarial scenario one can assume as it is hard for a malicious user to gain control over underlying network links.
Consequently, the scope for adversarial actions becomes restricted primarily to entities of substantial scale such as large corporations and governmental bodies.
In fact, Schuster\etalc{schuster2017beautyburst} show that an adversary, who executes JavaScript client code in the same browser as a victim, can accurately extract traffic pattern of the victim.
Pacer~\cite{mehta2022pacer} extends this threat model to cloud infrastructure, where the adversary deliberately colocates a malicious VM with the victim's VM.
Then, the adversary reveals the shape of the victim's traffic by observing the contention with the malicious VM's traffic.
To mitigate this attack, Pacer \cite{mehta2022pacer} proposes to use a cloaked tunnel.
Pacer's cloaked tunnel is conceptually similar to {\sys}'s shaping tunnel. 
Pacer precisely controls the transmit time of individual
TCP packets in accordance with the shaping schedule and congestion control
signals. 
However, {\sys} relies on post-processing property of DP to ensure that time and sizes of outgoing packets are differentially private. 
Thus, Pacer's tunnel endpoint requires tight integration with, and
non-trivial changes, to the network stack on the end hosts.
To shape traffic, Pacer splits a server's dataset into clusters of a minimum size $k$, while minimizing padding overhead for the dataset.
For each cluster, Pacer then computes a traffic shape based on network traces of all cluster objects.
As a result, within each cluster, traces are indistinguishable for the attacker. 
Nonetheless, similar to other clustering methods, the number of traces within a cluster influences the privacy of the traces it contains.


\section{Traffic Shaping without Pre-profiling}\label{subsec:dynamic-traffic-shaping}
In contrast to the previous approaches, these methods do not require prior profiling of application traces, and the shaping mechanism determines the shape of outgoing traffic directly at transmission time.
Shaping methods without pre-profiling are typically more practical compared to approaches that require profiling, as it can be challenging to profile all possible traffic patterns for sophisticated network applications.
However, ad-hoc decisions regarding the shaping of traffic at the time of transmission can potentially result in information leakage.
We provide an overview of four methods that do not require pre-profiling.
We start with an BuFLO~\cite{dyer2012peek} and its extension CS-BuFLO~\cite{cai2014csbuflo} in \Cref{subsubsec:buflo}. 
These two methods are relaxations of the constant shaping mechanism.
In \Cref{subsubsec:wtf-pad}, we also provide an overview WTF-PAD~\cite{juarez2016toward} as an effort to reduce the overhead of constant-shaping without providing a formal notion of privacy.
We discuss Tamaraw~\cite{cai2014systematic} and its privacy notion as one of few techniques that also provides a formal notion of privacy in \Cref{subsubsec:tamaraw}.
Finally, in \Cref{subsubsec:ditto}, we provide an overview of Ditto~\cite{meier2022ditto} as the state-of-the-art traffic shaping mechanism.  
For each mechanism, we highlight its unique characteristic and discuss its strengths and weaknesses.



\subsection{BuFLO and CS-BuFLO}\label{subsubsec:buflo}
Dyer et al.~\cite{dyer2012peek} conducted a comprehensive study on network side-channel attacks and state-of-the-art defense mechanisms available at the time, providing valuable insight into this field. 
Their negative results showed that none of the countermeasures at the time could completely mitigate network side-channel attacks.
To address this problem, they proposed a shaping mechanism known as Buffered Fixed-Length Obfuscator (BuFLO).
BuFLO can be considered a relaxation of the constant shaping method.
For every given webpage $W$, BuFLO sends fixed-sized packets at constant intervals for a specific duration of time.
If the packet sequence of webpage $W$ takes longer than a threshold, $T$, BuFLO continues sending fixed-sized packets at fixed intervals until transmission is finished, revealing the duration of the flow. 
When all flows have durations shorter than $T$, BuFLO effectively operates similarly to constant shaping.
In such cases, BuFLO inherits the advantages and disadvantages associated with constant shaping methods.~\cite{sirinam2018df}

To address problems associated with BuFLO, Cai~\etal~\cite{cai2014cs} proposed Congestion Sensitive Buffered Fixed-Length Obfuscator (CS-BuFLO).
CS-BuFLO enhances the BuFLO traffic shaping method by applying it bidirectionally, including both the client-to-server and server-to-client directions. 
To optimize network latency and reduce network load, CS-BuFLO dynamically adjusts the frequency of packets at the server side based on the client's transmission rate.
Additionally, to address the issue of fixed transmission durations, CS-BuFLO rounds page sizes to the nearest power of two. 
Overall, CS-BuFLO is a more pragmatic approach for traffic shaping compared to BuFLO. 
However, it is important to note that both the transmission rate adjustment and the padding to powers of two in CS-BuFLO have the potential to leak information.
The authors of CS-BuFLO, however, do not provide a quantification of the extent of information leakage resulting from these mechanisms






\subsection{WTF-PAD}\label{subsubsec:wtf-pad}
WTF-PAD~\cite{juarez2016toward} is a simple generalization of the Adaptive Padding (AP)~\cite{shmatikov2006timing} method used in Tor~\cite{dingledine2004tor}. 
The core shaping mechanism in this method is the same as adaptive padding. 
In adaptive padding, traffic shaping works based on a state machine with three states: \textit{Wait}, \textit{Burst}, and \textit{Gap}. 
The shaping procedure starts in \textit{Wait} state. 
Upon receiving a packet, shaping mechanism state changes to \textit{Burst} mode. 
WTF-PAD measures the inter-arrival time of the next packets. 
If the inter-arrival time is less than a threshold defined in algorithm, it remains in \textit{Burst} state. Otherwise, the state changes to \textit{Gap}.
In \textit{Gap} state, WTF-PAD samples a random variable from a distribution of inter-arrival times for packets during a traffic burst.
This sampled random variable serves as a timer, determining the interval at which the next dummy packet should be transmitted.
When an application sends a packet, the shaping mechanism transitions to the \textit{Burst} state. Otherwise, the mechanism remains in the \textit{Gap} state, continuing to send dummy packets at random intervals. 

Although WTF-PAD pad has zero latency overhead and moderate bandwidth overheads, it provides no formal privacy guarantees.
In fact, multiple new traffic analysis attacks are able to extract users information from their traffic pattern while WTF-PAD is deployed~\cite{sirinam2018df}.


\subsection{Tamaraw}\label{subsubsec:tamaraw}
Cai et al.~\cite{cai2014systematic} conducted a systematic study on network side-channel defenses, proposing a novel notion for privacy in defense mechanisms.
Additionally, they developed a network side-channel defense based on BuFLO~\cite{dyer2012peek} called Tamaraw. 

We summarized their notion of privacy in \Cref{sec:tamaraw-dp-comparison}. Here, we discuss the Tamaraw defense mechanism.



% To disambiguate with {\sys}'s notion of $(\varepsilon, \delta)$-DP, we rename Tamaraw's $\epsilon$ variable with $\gamma$ in this section.
% We show that $(\varepsilon, \delta_{\winlen})$-DP definition is strictly stronger than Tamaraw's $\gamma$-security definition.


Tamaraw's shaping method is almost the same as BuFLO~\cite{dyer2012peek}. The only difference it that instead of padding the total transmission time a fixed length, Tamaraw pads the total number of packets to multiples of a padding parameter $L$.
In other words, if the total number of packets for a webpage $W$ is $n_W$, Tamaraw pads it to a size of $\lceil \frac{n_W}{L} \rceil \times L$.
Increasing $L$ increases Tamaraw' privacy but this results in higher overheads.


\subsection{Ditto}\label{subsubsec:ditto}
The problem of network side channels is well-studied in computer networks.
The focus of the community, however, was mainly on Local Area Networks (LAN) and not on the Wide Area Networks (WAN).
Many large organizations, such as Google, Amazon, and Microsoft, rely on a WAN to connect their data centers.
This makes WANs an attractive target for eavesdropping and traffic analysis attacks, as they often transmit sensitive data.
Techniques that are proposed to address traffic analysis attacks, such as BuFLO~\cite{cai2014cs} and Pacer~\cite{mehta2022pacer}, are designed for Internet applications.
These methods often require modification on software and protocols of end hosts, which is challenging for many large organizations, because it is impossible for cloud providers to modify the software running on a customer instance.
Besides that, WAN traffic is high-throughput in nature, and many proposed solutions are not efficient enough to be used in this context.

Ditto~\cite{meier2022ditto} is a traffic shaping mechanism designed specifically for WAN.
Ditto performs traffic shaping in a programmable switch at the packet-level.
As for the shaping strategy, Ditto sends packets at fixed intervals.
To obfuscate packet sizes, Ditto defines a constant sequence of packet sizes $[L_1, L_2, \dots, L_N]$ and defines a state machine based on packet sizes.
When the shaping mechanism is in state $L_i$, Ditto sends a packet with the size of $L_i$.
If there is a packet in switch queues with a size $L < L_i$, Ditto pads the packet to size $L_i$ and transmits it.
Otherwise, it sends a chaff/dummy packet with size $L_i$.
This guarantees that the size of packet is independent of the input.
Ditto moves between packet states (the size of the next outgoing packet), based on the distribution of overall packet sizes.
For example, if 25\% of packets have size $X$ and 75\% have the size $Y$, Ditto will send the packet sequence of $[X, Y, Y, Y]$.

Ditto adds small latency to users' traffic, as it always transmits data.
However, the bandwidth overhead of the mechanism can potentially become large, as the shape of outgoing traffic is independent of the shape of input.
For example, if the packet sequence defined by Ditto has only one element, the Ditto shaping mechanism is the same as constant shaping.  

\section{Adversarial Noise}\label{subsec:adversarial-noise}
Advances in Machine Learning in general, and Deep Neural Networks (DNNs) in particular, have amplified the inferential capabilities of adversaries in network side-channel attacks~\cite{sirinam2018df}.
Although deep neural networks are powerful in mapping traffic traces of users to their content, they are vulnerable to adversarial examples~\cite{goodfellow2014explaining}.
Adversarial noise involves introducing perturbations to the original input provided to a classifier, leading the classifier to incorrectly classify the input into classes other than the correct one.
Adversarial noise can be used to defeat ML-based network side-channel attacks.
A defense mechanism against network side-channel attacks, constructed using adversarial noise, strategically adds noise with a specific pattern into traffic traces, thereby causing the attacker's model to misclassify them.
Now, we provide an overview of the state-of-the-art defense mechanism based on adversarial noise, WF-GAN. 
Unlike {\sys}, these methods are tailored to a specific classifier, limiting their generalization.
Besides that, adversarial noise does not provide any form of theoretical guarantees.

\subsection{WF-GAN}\label{subsubsec:wf-gan}
WF-GAN~\cite{hou2020wf} has three main components: Generator, Discriminator, and Target Model.
Generator receives a traffic trace, $x$, from the website $W$ and generates adversarial noise, $G(x)$. 
WF-GAN adds the output of the Generator to the input trace, creating an adversarial trace that resembles the traffic trace of another website $W'$.
We represent this trace as $x'$ such that $x'=x + G(x)$.
Then, Discriminator takes the original trace, $x$, and the corresponding adversarial trace, $x'$, as inputs and decides whether an input is a trace of website $W$ or website $W'$.
Therefore, Generator and Discriminator progress together; the former in generating additive noise that can cause traces of $W$ to be classified as $W'$ and the latter in distinguishing these two from each other.
Additionally, WF-GAN uses a state-of-the-art trained DNN model for traffic analysis in the last stage of its pipeline, where the DNN receives the output of the Generator $x'$ and decides whether it is an adversarial traffic trace or not. 
The goal is to create an adversarial trace such that the state-of-the-art model cannot distinguish it from non-adversarial traces.

WF-GAN does not provide any form of theoretical guarantees. 
Moreover, relying on one trained model to adjust the output traffic shape limits the defense's effectiveness to the characteristics of that particular model only.
If the attacker changes its model, the defense mechanism must either be retrained or risk its ability to maintain consistent levels of effectiveness against different attacks.
Finally, adversarial noise defense mechanisms are shown to be vulnerable against adversarial training, a technique wherein the attacker trains their model on a combination of adversarial and original data to effectively differentiate between them~\cite{zhang2019statistical}.


\section{Differential Privacy}\label{sec:dp-background}
% Subsection intro. Elaborate on origins of DP. Why it is useful.
Data scientists always strive to understand the general properties of a population. 
To answer questions concerning the etiology of a disease, factors contributing to a societal phenomenon, or the consequences of an economic policy, researchers collect data from individual people. They then process this data to calculate population-level statistics and propose solutions based on these aggregated statistics.
Ensuring the privacy of individuals who have contributed to these studies is of utmost importance from both ethical and legal perspectives.
In fact, the intended goal of all scientific studies is to collect valuable information about the targeted population not individual people within the population. 
However, it is impossible to learn useful information about a population while learning \textit{nothing} about individuals, leading to a paradox between usefulness of a dataset (\ie utility) and its privacy. 
Differential Privacy (DP) addresses this paradox by quantifying the extent of privacy leakage for individuals in a dataset when publishing its statistics.
Within the framework of Differential Privacy (DP), the aforementioned paradox transforms into an adjustable trade-off between the utility of data and the preservation of privacy. 

In the rest of this section, we provide the formal definition of differential privacy, highlight its main properties, and explore its potential application in the domain of traffic shaping.


%
%% 
%% Definition
%%
%


\subsection{Definitions}\label{subsec:background-dp-definitions}
A database $D$ has a fixed number of entries, where each individual data point is stored in one entry.
Following the notation proposed by Dwork \etal \cite{dwork2014algorithmic}, we represent databases by their histograms: $D \in \dbSpace$. 
In this representation, every entry $D_i$ of the database $D$ represents the number elements of type $i \in \mathcal{X}$.
\\  
Within the DP context, we define a \textit{query} as a function that operates on a database. 
Any computation performed on a database, regardless of its codomain, can be regarded as a query executed on that database.
The primary objective of differential privacy is to offer meaningful population-level information for queries executed on a database, all while protecting the privacy of individual data points within the database.
Intuitively, small changes in a database should not significantly impact the outcome of a query.
To further clarify the terms "small changes" and "significant impact", we provide definitions for neighboring databases and query function sensitivity, respectively.
\\
We measure the difference between two databases with a distance metric $\rho(D, D')$.
\begin{definition}[Neighboring databases]
  Given a distance metric $\rho$, we refer to two databases $D, D' \in \dbSpace$ as neighboring databases, if and only if $\rho(D, D') \leq 1$.             
\end{definition}
\noindent In the standard DP definition, the distance metric is simply the number of different data entries in two databases (\ie Hamming distance).
However, Chatzikokolakis \etal \cite{chatzikokolakis2013broadening} show that DP applies to a general definition of distance metrics.
\\
To quantify the extent to which a single data point can influence the outcome of a query in worst-case, we introduce the concept of sensitivity. 
\begin{definition}[$l_p$-Sensitivity]
  \label{def:norm-sensitivity}
  The $l_p$-sensitivity of a query function $f$ is:
  \begin{equation*}
    \Delta_p f = \max_{D, D'} \|f(D) - f(D')\|_p 
  \end{equation*}
where $\rho(D, D') \leq 1$ (\ie $D$ and $D'$ are neighboring databases).
\end{definition}
\noindent
Differential Privacy involves randomization of query outputs. We define a randomized algorithm as follows.
\begin{definition}[Randomized algorithm]
  \label{def:randomized-algorithm}
  A randomized algorithm $M$ with the domain $A$ and discrete range $B$, for any given $a \in A$ and $b \in B$ outputs $M(a)= b$ with probability of $(M(a))_b$.
\end{definition}

With all the necessary components in place, we are now prepared to present the formal definition of differential privacy.
\begin{definition}[Differential privacy]
  \label{def:dp}
  A randomized algorithm $M: \dbSpace \rightarrow \mathbb{R}$ is $(\varepsilon, \delta)$-differentially private if for all ${S} \subseteq Range(M)$ and for  all $D, D' \in \dbSpace$ such that $\rho(D, D') \leq 1$, we have:
  \begin{equation*}
    \Pr[M(D) \in S] \leq \exp(\varepsilon)\Pr[M(D') \in S] + \delta
  \end{equation*}
\end{definition}
\noindent
Indeed, Differential Privacy (DP) is a definition rather than a specific algorithm.
It provides a framework for ensuring privacy guarantees in various randomized algorithms. Multiple randomized algorithms can achieve $(\varepsilon, \delta)$-privacy for a given set of databases, each with different characteristics. 
Intuitively, given an output, with probability $1-\delta$ the log likelihood ratio of running the algorithm $M$ on databases $D$ and $D'$ is bounded by $\varepsilon$.
This bound ensures that the presence or absence of any individual's data in the database has a limited impact on the likelihood of obtaining a particular output.
The smaller $\varepsilon$ implies that both neighboring databases are equally likely to generate the output, resulting in more privacy.
The parameter $\varepsilon$ is commonly referred to as \textit{privacy loss} within the context of DP.
The parameter $\delta$ determines the failure probability of a differential private mechanism and is typically expected to have a small value (\ie smaller than $10^{-5}$).


%
%% 
%% Properties
%%
%

\subsection{Properties}\label{subsec:background-dp-properties}
In this section, we explore the key properties of Differential Privacy as a privacy framework.


% ROBUSTNESS TO AUXILIARY INFORMATION.
\subsubsection{Robustness to auxiliary information}
\label{subsubsec:dp-auxiliary}
The definition of Differential Privacy does not make any assumptions regarding the prior knowledge of the adversary. 
In other words, regardless of the adversary's prior knowledge, the information gained by the adversary after observing the output of a differentially private algorithm $M$ remains within the bounds specified by Differential Privacy. 
\begin{proposition}
  \label{prop:auxiliary}
  Assume that the adversary has a prior $\Pr(D)$ over the set of all possible databases $D, D' \in \dbSpace$, given the output $S$ of a $(\varepsilon, \delta)$-DP algorithm $M: \dbSpace \rightarrow \mathbb{R}$, for all $D, D' \in \dbSpace$ such that $\rho(D, D') \leq 1$, we have: 
  \begin{equation*}
    \frac{\Pr(D|S)}{\Pr(D'|S)} \leq \exp(\varepsilon) \frac{\Pr(D)}{\Pr(D')}
  \end{equation*}
\end{proposition}
In broad terms, robustness to auxiliary information in the context of Differential Privacy is similar to the security semantics of cryptographic algorithms. 
For instance, consider a scenario where an adversary possesses the knowledge that the content of an encrypted message is either a picture of a car or a picture of a tree.
In this case, observing the encrypted message does not provide any additional evidence to indicate which of the two possibilities is more likely to be the true message.
In DP, nevertheless, observing the results of private queries does change the prior knowledge of the adversary.
However, this change remains within the boundaries defined by DP and does not exceed them.

% POST-PROCESSING
\subsubsection{Post-processing}
\label{subsubsec:background-dp-postprocessing}
Differential Privacy guarantees are resilient to post-processing of the output from a DP algorithm.
In fact, in the absence of any additional knowledge, the adversary is unable to undermine the guarantees of Differential Privacy simply by processing the output of the algorithm.
\begin{proposition}
  \label{prop:post-processing}
  Consider $M: \dbSpace \rightarrow R$ as a $(\varepsilon, \delta)$-differentially private algorithm, and let $g: \mathbb{R} \rightarrow \mathbb{R}$ be an arbitrary randomized mapping, then $g(M(.)): \dbSpace \rightarrow \mathbb{R}$ is $(\varepsilon, \delta)$-differentially private. 
\end{proposition}
\noindent
This implies that regardless of the complexity of a procedure, as long as the inputs are differentially private, the output is guaranteed to maintain the same level privacy.
The post-processing property of Differential Privacy can be especially advantageous when dealing with systems that have information bottlenecks, such as situations where results from multiple calculations are aggregated at a single stage.
In such systems, by applying a differentially private algorithm to the information bottleneck, the guarantee of Differential Privacy extends to the output of the entire system. 
We specifically utilize post-processing property of Differential Privacy in the design of both the {\sys} shaping mechanism and the {\sys} middlebox. 

% PRESERVATION UNDER ADAPTIVE SEQUENTIAL COMPOSITION.
\subsubsection{Self-Composition}\label{subsubsec:background-dp-composition}
The simultaneous release of results from multiple differentially private algorithms maintains the differential privacy guarantee.
This property facilitates the modular construction of differentially private algorithms, allowing multiple DP algorithms to be combined to create more sophisticated and advanced algorithms.
Furthermore, the composition theorem enables us to compute the privacy parameters associated with the sequential release of a differentially private algorithm output, thereby facilitating multiple releases of the same DP output.
There are multiple variants of composition theorem, and we with the simplest form of composition.
\begin{proposition}[Basic composition theorem]
\label{prop:basic-composition}
  Let $M_i: \dbSpace \rightarrow \mathbb{R}$ be an $(\varepsilon_i, \delta)$-differentially private algorithm for $i \in [k]$. Then, the combination of these $k$ algorithms, $M_{[k]}: \dbSpace \rightarrow \Pi_{i=1}^{k}\mathbb{R}$ is $(\sum_{i=1}^{k}\varepsilon_i, \sum_{i=1}^{k}\delta_i)$-differentially private.  
\end{proposition}
Basic composition theorem states that the privacy loss resulting from the combination of multiple differentially private algorithms is equal to the aggregate of their individual privacy losses.
While functional, the basic composition theorem tends to overestimate the privacy loss of combined DP algorithms.
The advanced composition theorem offers a more precise and rigorous bound for the aggregated privacy loss incurred by multiple differentially private algorithms. 
\begin{proposition}[Advanced composition theorem]
\label{prop:advanced-composition}
  Let $M_i: \dbSpace \rightarrow \mathbb{R}$ be an $(\varepsilon, \delta)$-differentially private algorithm for $i \in [k]$. Then, the combination of these $k$ algorithms, $M_{[k]}: \dbSpace \rightarrow \Pi_{i=1}^{k}\mathbb{R}$ is $(\varepsilon', k\delta+\delta')$-differentially such that:
  \begin{equation*}
    \forall \delta' \geq 0: \varepsilon' = \sqrt{2k\ln(1/\delta')}\varepsilon + k\varepsilon(e^{\varepsilon} - 1)
  \end{equation*}
\end{proposition}
It is important to note that this form of composition theorem is only applicable to differentially private algorithms that share the same privacy parameters.
More sophisticated versions of the composition theorem exist, offering tighter privacy bounds~\cite{kairouz2015composition, mironov2017renyi}.
We particularly use R{\'e}nyi Differential privacy to calculate the privacy loss of our differentially private shaping mechanism.

\subsection{Mechanisms}\label{subsec:background-dp-mechanism}
\Cref{def:randomized-algorithm} provides a general definition of a randomized algorithm, while \Cref{def:dp} outlines the specific criteria that a randomized algorithm must satisfy to be considered differentially private (\ie DP mechanism).
In this section, we will explore the two most commonly used differentially private mechanisms: the Laplace mechanism and the Gaussian mechanism\cite{dwork2014algorithmic}.
\begin{definition}[Laplace Mechanism]\label{def:laplace-mechanism}
  Given any function $f: \dbSpace \rightarrow \mathbb{R}^k$ the Laplace mechanism is defined as:
  \begin{equation*}
    \mathcal{M}(x, f, \varepsilon) = f(x) + (Y_1, Y_2, \dots, Y_k)
  \end{equation*}
  where $Y_i$ are i.i.d random variables from Laplace distribution with probability density function of $Lap(\Delta_1 f/\varepsilon)$, and $\Delta_1 f$ is $l_1$-sensitivity of function $f$ (see \Cref{def:norm-sensitivity}).
\end{definition}
The Laplace mechanism is relatively easy and straightforward. 
Intuitively, it just adds a random variable driven from a Laplace distribution to the result of the query.
The variance of this random variable quantifies the privacy of query results, as a higher variance makes it more challenging for adversaries to infer the true result of the query.
\begin{proposition}
  The Laplace mechanism of \Cref{def:laplace-mechanism} is $(\varepsilon, 0)$-differentially private. 
\end{proposition}
\noindent
We omit the proof here; you can refer to the work of Dwork\etalc{dwork2014algorithmic} for a detailed proof. 
As we can see in \Cref{def:laplace-mechanism}, the failure probability of Laplace mechanism, $\delta$, is 0.
In broad terms, a mechanism can compromise failure rate $\delta$ to add less noise to query results while achieving same values for privacy loss $\varepsilon$.
\begin{definition}[Gaussian Mechanism]\label{def:gaussian-mechanism}
  Given any function $f: \dbSpace \rightarrow \mathbb{R}^k$ the Gaussian mechanism is defined as:
  \begin{equation*}
    \mathcal{M}(x, f, \varepsilon, \delta) = f(x) + (Y_1, Y_2, \dots, Y_k)
  \end{equation*}
  where $Y_i$ are i.i.d random variables from Gaussian distribution $\mathcal{N}(0, \frac{2\Delta_2 f^2}{\varepsilon^2})\ln(\frac{1.25}{\delta})$, and $\Delta_2 f$ is $l_2$-sensitivity of function $f$ (see \Cref{def:norm-sensitivity}).
\end{definition}
We particularly design our differentially private traffic shaping mechanism using Gaussian mechanism at its core. 
We elaborate on this in \Cref{subsec:dp-shaping-mechanism}.

\subsection{R\'enyi Differential Privacy}
Despite widespread usage, straightforward interpretability,, and numerous applications of standard definition of $(\varepsilon, \delta)$-differential privacy, this concept does exhibit two primary limitations. 
First, the presence of the failure probability $\delta$ in this definition contradicts the guarantee of  plausible deniability associated with differential privacy since with probability $\delta$ the secret can be completely exposed.    
Secondly, as we mentioned in \Cref{subsec:background-dp-properties}, the desirable results of strong composition theorem only holds for homogeneous DP mechanisms.
In fact, Vadhan~\etalc{murtagh2015complexity} demonstrate that the generalization of advanced composition theorem to heterogeneous DP mechanisms (\ie $(\varepsilon_i, \delta_i)$-DP for different values of $\varepsilon_i$ and $\delta_i$) is P-hard.
To address these limitations, Mironov~\etalc{mironov2017renyi} propose a new definition of differential privacy based on the R\'enyi divergence~\cite{renyi1961measures}.
R\'enyi differential privacy (RDP) is strictly stronger than $\varepsilon, \delta$-DP, meaning that any $\varepsilon, \delta$-DP mechanism is RDP though the reverse does not hold true.
In this section, we explain the core concepts of R\'enyi differential privacy (RDP), elaborate on its properties, and highlight its connection to standard definition of DP.

\begin{definition}[R\'enyi divergence]
  for two probability distributions $P$ and $Q$, the R\'enyi divergence of order $\alpha > 1$ is defined as:
  \begin{equation*}
    D_{\alpha}(P||Q) \triangleq \frac{1}{\alpha-1} \log E_{x \sim Q}\left ( \frac{P(x)}{Q(x)} \right )
  \end{equation*}
\end{definition}
R\'enyi divergence simply measure the difference between two distributions $P$ and $Q$.
For $\alpha = \infty$, the R\'enyi divergence specifically defined as:
\begin{equation*}
  D_{\infty}(P||Q) \triangleq  \sup_{x \sim Q} \; \log \left ( \frac{P(x)}{Q(x)} \right ) 
\end{equation*}
We can see that the R\'enyi divergence with $\alpha=\infty$ is closely connected to $(\varepsilon, 0)$-DP definition.
\begin{proposition}
  A randomized algorithm $M: \dbSpace \rightarrow \mathbb{R}$ is $(\varepsilon, 0)$-differentially private iff for all $D, D' \in \dbSpace$ such that $\rho(D, D') \leq 1$, we have:
  \begin{equation*}
    D_{\infty}(P||Q) \leq \varepsilon
  \end{equation*} 
\end{proposition}
\noindent Next, we provide the formal definition of R\'enyi DP.
\begin{definition}[$(\alpha, \varepsilon)$-RDP]
  A randomized algorithm $M: \dbSpace \rightarrow \mathbb{R}$ is $(\alpha, \varepsilon)$-RDP if for all $D, D' \in \dbSpace$ such that $\rho(D, D') \leq 1$, we have:
  \begin{equation*}
    D_{\alpha}(\mathcal{M}(D)||\mathcal{M}(D')) \leq \varepsilon
  \end{equation*}
\end{definition}
\noindent Indeed, all the favorable attributes of the standard definition of differential privacy, including post-processing, self-composition, and robustness to auxiliary information, remain applicable to R\'enyi differential  privacy.
Particularly, the composition theorem has a simple representation in RDP.
\begin{proposition}[R\'enyi Composition]\label{prop:rdp-composition}
  Let $M_i: \dbSpace \rightarrow \mathbb{R}$ be an $(\alpha, \varepsilon_i)$-RDP algorithm for $i \in [k]$. Then, the combination of these $k$ algorithms, $M_{[k]}: \dbSpace \rightarrow \Pi_{i=1}^{k}\mathbb{R}$ is $(\alpha, \sum_{i=1}^{k}\varepsilon_i)$-RDP.   
\end{proposition}
\noindent For an in-depth proof, we direct readers to the R\'enyi differential privacy paper~\cite{mironov2017renyi}.
So far, we explained three methods to calculate the aggregated privacy loss of multiple releases of differentially-private mechanism.
These methods are basic composition theorem, advanced combination theorem, and R\'enyi composition theorem.
The goal of these methods is to provide a tight, accurate bound on aggregated privacy loss without overestimation. 
In simpler terms, a composition theorem is superior to another, if it can provably show that the aggregated privacy loss of a differentially-private mechanism, after multiple release of results, is smaller than the aggregated privacy loss indicated by the other mechanism.
A natural question is: \textit{What is the difference between all these mechanisms?}
To answer these questions, we calculate the privacy loss reported by these mechanisms for different number of queries (\ie varying number of differentially-private releases of results).
We fix epsilon and delta per query $0.2$ and $0.0001$ respectively, and report the final privacy loss for a given number of queries.
\begin{figure}[t]
  \centering
  \includegraphics[width=\columnwidth]{plots/composition_comparison.pdf}
  \caption{Aggregated privacy loss of multiple queries reported by different composition methods}
  \label{fig:composition-comparison}
\end{figure}

\Cref{fig:composition-comparison} compares different methods for composing privacy loss of DP mechanisms.
We can see the effectiveness of R\'enyi methods as compared to others across all values for number of queries.
Advanced composition theorem also outperforms basic composition theorem for larger number of queries. 


As we mentioned earlier, $(\alpha, \varepsilon)$-RDP is strictly stronger than $\varepsilon, \delta$-DP. 
Next proposition shows how RDP implies standard DP.
\begin{proposition}\label{prop:rdp-better-than-dp}
  If $M: \dbSpace \rightarrow \mathbb{R}$ is an $(\alpha, \varepsilon)$-RDP mechanism, it is also $(\varepsilon + \frac{\log 1/\delta}{\alpha-1}, \delta)$-DP for any $0 < \delta < 1$. 
\end{proposition}
\noindent \Cref{prop:rdp-better-than-dp} enables us to interpret guarantees of R\'enyi differential privacy with semantics of standard differential privacy. 
Besides that, \Cref{prop:rdp-composition} provides us with the means to compose heterogeneous differentially private algorithms, while and at the end, with \Cref{prop:rdp-better-than-dp} we can still represent the final result using semantics of standard DP. 
In order to compute the privacy loss (\ie $\varepsilon$) of our differentially private shaping mechanism in \Cref{sec:dp-privacy-guarantees}, we use \Cref{prop:rdp-composition} referred to as $\textrm{DP\_compose}()$.

\section{QUIC Transport Protocol}\label{sec:background-quic}

QUIC is a modern transport layer protocol specifically developed to enhance the performance of HTTPS traffic~\cite{langley2017quic}.
Using UDP packets as the transport layer carrier allows QUIC packets to traverse current network infrastructure developed for TCP/UDP protocols such as middleboxes.
On the top of UDP, QUIC provides transport functionalities of flow control, loss recovery, encryption, and congestion control.
Unlike TCP, QUIC mitigates head-of-the-line blocking delays by introducing a novel data structuring abstraction known as streams. 
QUIC is now standardized under RFC 9000~\cite{rfc9000}, and the details of design and implementation of the protocol is well beyond the scope of this thesis.
QUIC's streams offer a lightweight and ordered way for applications to send byte-streams.
These streams can be either unidirectional or bidirectional, depending on the specific needs of the application.
Within a connection, each stream is identified with a new ID.
Each stream has a unique header with 5 major fields: Type, Stream ID, Offset, and Data Length.
The stream header within a packet is encrypted, ensuring that adversaries are unable to discover the active streams in a connection.
Either of endpoints can terminate streams.

% \subsection{QUIC Congestion and Flow Control}
% \subsubsection{Flow Control}
% To prevent a fast or malicious sender from overwhelming receivers' buffer, receivers must proactively limit the amount of data a sender can send.
% QUIC can perform flow control at two levels: The receiver controls the maximum amount of data the sender can send on a per-stream basis, as well as across all streams within a connection. 
% The receiver sets the initial limits for stream and connection flow controls during the handshake. 
% During the handshake, the receiver establishes the initial limits for both stream and connection flow controls. Throughout the connection, if the receiver needs to adjust the limits for a specific stream, it sends \texttt{MAX\_STREAM\_DATA} frames to the sender.
% Similarly, to change the limit for overall connection, the receiver sends \texttt{MAX\_DATA} frames to the sender.

% \subsubsection{Congestion Control}
% QUIC does not standardize any specific method of congestion control.
% Instead, it provides the necessary feedback mechanism to implement congestion control.
% The modular design of congestion control makes QUIC more flexible, allowing applications to implement a congestion mechanism that aligns with their specific requirements.
% As the default congestion control mechanism, however, QUIC uses a sender specified congestion controller similar to TCP NewReno~\cite{rfc6582}.
% Given the limitations of this thesis, a comprehensive explanation of TCP NewReno is outside its scope.
% For a more in-depth understanding of this mechanism, we kindly refer readers to the RFC, which provides detailed information and insights.







\section{Threat Model}\label{sec:threat-model}
We assume that the communicating parties reside in separate trusted private networks (\eg each node is behind a VPN gateway node) that an adversary cannot compromise.
We assume that all the application endpoints are non-malicious and do not leak the secrets themselves.
The adversary with control over the underlying network links (e.g., Internet Service Providers) can monitor, manipulate, and record the victim's traffic pattern.
In particular, it can precisely record the traffic shape---the sizes, timing, and direction of packets---on all links in the victim's traffic communication graph.
Furthermore, the adversary can drop, replay, or inject packets into the victim's traffic.
However, it cannot break standard cryptography, cannot compromise the victim's VPN, and cannot impersonate its clients or servers to directly interact with the victim.
We do not consider threats due to observing the IP addresses of packets.
We also do not consider the threat where a victim accidentally installs a malicious script in the browser, thus enabling an adversary to colocate with the
victim's application and observe its traffic~\cite{schuster2017beautyburst,mehta2022pacer}.
This is a reasonable assumption, since a colocated adversary can exploit many other direct or indirect channels for data leaks that will be far more efficient than {\nsc}s~\cite{kocher2018spectre, yarom2014flushreload, liu2015llcpractical, irazoqui2015ssa, vila2017loophole}.
{\sys}'s trusted computing base (TCB) includes all components in the organization's private network and the implementation of {\sys} tunnel endpoint.
We that assume the cryptographic libraries are side-channel free~\cite{almeida2016verifying}.

{\sys} does not address leaks of one application's sensitive data through the traffic shape of colocated applications transmitting only non-sensitive traffic.
Such leaks may arise, for instance, due to microarchitectural interference among the applications colocated on a host or among their flows if they pass through shared links.
Mitigating such leaks would require physically isolating the
applications and their flows.
For instance, a service serving both privacy-sensitive and privacy-insensitive clients could be partitioned into two instances, with each instance dedicated to serving clients with similar privacy requirements.
For the privacy-sensitive clients, we assume that the corresponding service instance agrees to serve the clients according to their privacy requirements. 
Alternatively, end hosts could implement mitigations for various side channels to support colocated applications \cite{mehta2022pacer, page05partitionedcache, shi2011limiting, kim2012stealthmem, varadarajan2014scheduler, braun2015robust}. 
To mitigate leaks among colocated flows at shared links, {\sys} assumes that the network paths of the sensitive and non-sensitive flows are completely partitioned.
In practice, this limitation can be removed by combining {\sys}’s traffic shaping with TDMA scheduling \cite{beams2021ifs, vattikonda2012tdma}.


Under these assumptions, {\sys} prevents leaks of application secrets through the sizes and the delays between the network packets transmitted in either direction between the application endpoints.






