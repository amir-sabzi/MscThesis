% Section Intro, elaborate on the general theme of the section. 
In this section, we explore previous mitigation approaches proposed to address network side-channel attacks. 
We categorize these approaches into 4 general categories: 1- Static shaping methods, 2- Dynamic shaping methods, 3- Hybrid traffic shaping methods, 4- Adversarial traffic patterns. 




\subsection{Static Traffic Shaping}\label{subsec:static-traffic-shaping}
Static traffic shaping methods typically consist of two distinct phases: the profiling phase and the shaping phase.
During the profiling phase, the shaping mechanism collects user-specific information from the user's traffic traces.
This information can vary and may include details such as the exact traffic pattern, distribution of packet sizes, or timing of packet transmission.
The shaping mechanism then leverages this information during the shaping phase to transform users' traffic into a new pattern that ensures privacy with minimal overhead.
However, scaling static traffic shaping methods is challenging, or even impossible in many cases, due to the need for profiling numerous traffic traces.
% methods we are going to cover:
%% Walkie-Talkie 
%% Supersequence
%% Glove
%% Traffic morphing 
\subsubsection{Traffic Morphing}\label{subsubsec:traffic-morphing}
Traffic morphing~\cite{wright2009traffic} is a static approach for traffic shaping. 
In this method, the shaping mechanism alters the distribution of packet sizes in such a way that they resemble packets generated by a different website rather than the original website.
This helps to obfuscate the traffic and make it harder for eavesdroppers to associate the packets with their true source.
Traffic Morphing (TM) operates based on a technique known as \textit{Direct Target Sampling}.

First it collects the distribution of packets sizes for all websites in the dataset.
Then, Given a webpage $W$, TM randomly selects another webpage $W'$ from the dataset.
When selecting the target webpage $W'$, TM employs an optimization method to choose webpages that result in minimal overheads. 
Upon receiving each packet produced by $W$ with the size $l$, TM samples a packet size $l'$ from the distribution of $W'$. 
If $l' > l$, TM pads the outgoing packet with dummy data to match the size $l'$. 
On the other hand, if $l' < l$, TM sends $l'$ bytes and continues sampling from the distribution of $W'$ until all bytes of the original packet have been transmitted.
It is important to emphasize that the traffic morphing method does not hide the timing or duration of bursts of traffic.
Therefore, an adversary can still exploit these features of traffic traces to potentially reveal users' information based on their traffic patterns, even when traffic morphing is deployed~\cite{dyer2012peek}.


\subsubsection{Walkie-Talkie}\label{subsubsec:walkie-talkie}
Wang\etalc{wang2017walkie} propose \textit{Walkie-Talkie} as a new defense mechanism implemented in browsers to hide traffic shape of sensitive websites.
\textit{Walkie-Talkie} requires browsers modification to change the default full-duplex communication to half-duplex mode.
In full-duplex communication, multiple servers are actively transmitting web page data to the client, while the client concurrently submits additional resource requests, potentially to different servers.
In half duplex mode, on the other hand, the client only sends requests after the web servers have satisfied all previous requests.
Using half-duplex mode enables \textit{Walkie-Talkie} to change communication to a sequence of bursts of data between one client and one server at a time. 
\textit{Walkie-Talkie} represents each burst sequence $s$ as follows:
\begin{equation*}
  s = \{(b_{1+}, b_{1-}), (b_{2+}, b_{2-}), \dots \}
\end{equation*}
where $b_{i+}$ represents the size for the $i$th outgoing burst, and $b_{i-}$ represents the size for the $i$th incoming burst for sequence $s$.
\\
For any sensitive website $W$, \textit{Walkie-Talkie} first extract the burst sequence $s$.  
Then, using a similar approach to Traffic Morphing~\cite{wright2009traffic}, \textit{Walkie-Talkie} chooses a decoy website $W'$ with burst sequence of $s'$.
At the transmission time, \textit{Walkie-Talkie} reshapes the burst sequence of $s$ to sequence $\hat{s}$ such that:
\begin{equation*}
  (\hat{b}_{i+}, \hat{b}_{i-}) = (\max(b_{i+}, b_{i+}'), \max(b_{i-}, b_{i-}'))
\end{equation*}
where $\hat{b}_{i+}$, $b_{i+}$, and $b_{i+}'$ represent the size for the $i$th outgoing burst in sequences of shaped traffic $\hat{s}$, original webpage $s$, and decoy webpage $s'$ respectively.
The same notation used for incoming burst sizes.
Therefore, if an adversary observes the sequence $\hat{s}$, it can not ascertain whether the users accessed $W$ or $W'$.
Choosing multiple decoy burst sequences for any sensitive can further increase the privacy of \textit{Walkie-Talkie}.

Essentially, \textit{Walkie-Talkie} use as a clustering technique for shaping traffic, conceptually similar to Supersequence~\cite{wang2014supersequence}, Glove~\cite{nithyanand2014glove}, and Tamaraw~\cite{cai2014systematic}.
It maps the traffic shapes of various web pages to the same pattern, effectively rendering them indistinguishable for potential adversaries.
The advantages of the \textit{Walkie-Talkie} are: it adds small overheads compared to other defenses, requires minimal computation to extract the burst sequence of shaped traffic, and stores small metadata to for decoy webpages.
On the other hand, it had several disadvantages. 
First, \textit{Walkie-Talkie} requires browser modification.  
Secondly, similar to most of the static traffic shaping methods, \textit{Walkie-Talkie} should identify the burst sequence of a webpage before its transmission.
Finally, the adoption of half-duplex mode in browsers imposes a notable constraint on the scalability of this approach, as numerous applications, such as video streaming and file downloading, require multiple and simultaneous communications with web servers.


% ---------------------------------------------------% 
% ---------------------------------------------------% 
% ---------------------------------------------------% 
% ---------------------------------------------------% 
% ---------------------------------------------------% 

\subsection{Dynamic Traffic Shaping}\label{subsec:dynamic-traffic-shaping}
In these methods, the traffic shaping defense mechanism does not require prior profiling of application traffic traces in order to perform its function.  
In other words, the shaping mechanism determines the shape of outgoing traffic at the transmission time.
These methods are typically more practical compared to static approaches, as it can be challenging to profile all possible traffic patterns for sophisticated network applications.
On the other hand, ad-hoc decisions regarding the shaping of traffic at the time of transmission can potentially result in information leakage.
In this section, we provide an overview of dynamic traffic shaping methods, highlight their unique characteristics, and discuss their strengths and weaknesses.

\subsubsection{WTF-PAD}\label{subsubsec:wtf-pad}
WTF-PAD~\cite{juarez2016toward} is a simple generalization of Adaptive Padding (AP)~\cite{shmatikov2006timing} method to be used in Tor~\cite{dingledine2004tor}. 
The core shaping mechanism in this method is the same as adaptive padding. 
In this method, traffic shaping works based on a state machine with three states: \textit{Wait}, \textit{Burst}, and \textit{Gap}. 
The shaping procedure starts in \textit{Wait} state. 
Upon receiving a packet, shaping mechanism state changes to \textit{Burst} mode. 
WTF-PAD measures the inter-arrival time of the next packets. 
If the inter-arrival time is less than a threshold defined in algorithm, it remains in \textit{Burst} state. Otherwise, the states changes to \textit{Gap}.
In \textit{Gap} state, WTF-PAD samples a random variable from a distribution of inter-arrival times for packets during traffic burst.
This sampled random variable serves as a timer, determining the interval at which the next dummy packet should be transmitted.
When an application sends a packet, the shaping mechanism transitions to the \textit{Burst} state. Otherwise, the mechanism remains in the \textit{Gap} state, continuing to send dummy packets at random intervals. 

Although WTF-PAD pad has zero latency overhead and moderate bandwidth overheads, it provides no formal privacy guarantees.
In fact, multiple new traffic analysis attacks are able to extract users information from their traffic pattern while WTF-PAD is deployed.
\todo{Add the references and also mention attacks that are successful against WTF-PAD}.


\subsubsection{BuFLO and CS-BuFLO}\label{subsubsec:buflo}
Dyer et al.~\cite{dyer2012peek} conducted a comprehensive study on network side-channel attacks and state-of-the-art defense mechanisms available at the time, providing valuable insights into this field. 
Their negative results showed that none of the countermeasures at the time could completely mitigate network side-channel attacks.
To address this problem, they proposed a shaping mechanism known as Buffered Fixed-Length Obfuscator (BuFLO).
BuFLO can be considered as a relaxation of the constant shaping method.
For every given webpage $W$, BuFLO sends fixed-sized packets at constant intervals for a specific duration of time.
We represent packet sizes with $p$, packet frequency with $f$, and the sending duration with $T$. 
If the packet sequence of webpage $W$ takes longer than the specified time $T$, BuFLO continues sending fixed-sized packets at fixed intervals until transmission is finished, revealing the duration of the flow. 
When all flows have durations shorter than $T$, BuFLO effectively operates similarly to constant shaping.
In such cases, BuFLO inherits the advantages and disadvantages associated with the constant shaping method.
\todo{Add the references and also mention attacks that are successful against WTF-PAD}.

To address problems associated with BuFLO, Cai~\etal~\cite{cai2014cs} proposed Congestion Sensitive Buffered Fixed-Length Obfuscator (CS-BuFLO).
CS-BuFLO enhances the BuFLO traffic shaping method by applying it bidirectionally, encompassing both the client-to-server and server-to-client directions. 
To optimize network latency and reduce network load, CS-BuFLO dynamically adjusts the frequency of packets at the server side based on the client's transmission rate.
Additionally, to address the issue of fixed transmission durations, CS-BuFLO rounds page sizes to the nearest power of two. 
Overall, CS-BuFLO is a more pragmatic approach to perform traffic shaping compared to BuFLO. 
However, it is important to note that both the transmission rate adjustment and the padding to powers of two in CS-BuFLO have the potential to leak information.
The authors of CS-BuFLO, however, do not provide a quantification of the extent of information leakage resulting from these mechanisms


\subsubsection{Tamaraw}\label{subsubsec:tamaraw}
Cai et al.~\cite{cai2014systematic} conducted a systematic study on network side-channel defenses, proposing a novel notion for privacy in defense mechanisms.
Additionally, they developed a network side-channel defense based on BuFLO~\cite{dyer2012peek} called Tamaraw. 
We first overview their notion of privacy, and then, discuss Tamaraw defense mechanism.


Tamaraw \cite{cai2014tamaraw} provides a
mathematical notion of privacy guarantee of a shaping strategy, called $\epsilon$-security.
To disambiguate with {\sys}'s notion of $(\varepsilon, \delta)$-DP, we rename Tamaraw's $\epsilon$ variable with $\gamma$ in this section.
We show that $(\varepsilon, \delta_{\winlen})$-DP definition is strictly stronger than Tamaraw's $\gamma$-security definition.
We start by explaining Tamaraw's definition.
\\
\noindent
Tamaraw defines $W$ as the random variable that represents the label of a
traffic trace.
For each traffic trace, $w$, let $T_{w}$ $T_{w}^{D}$ be the random variables representing
the packet trace of $w$ before and after applying shaping on $w$ respectively.
The distribution of $T_{w}^D$ encompasses all variations in observed
patterns of a trace $w$ resulting from both the defense mechanism and the
network, and the distribution of $T_{w}$ only captures the randomness added in the network.
The attacker can measure the distribution of $W$ and $T_{w}^{D}$ independently.
\\
\noindent
Upon observing a trace $t$ on the network, an optimal attack $A$, selects the
label that corresponds to the maximum likelihood of observing that trace.
\begin{equation*}
  A(t) = \argmax_{w}{\Pr[W=w]\Pr[T_{w}^{D}=t]}
\end{equation*}
For any attack $A$, we represent the probability that attack output the label $w_i$ with $\Pr_A[w_i]$.
\begin{definition}[Tamaraw $\gamma$-privacy]
  A fingerprinting defense $D$ is said to be uniformly $\gamma$-private if for the attack $\mathcal{A}$ if we have:
  \begin{equation*}
    \max_w\big[\Pr[A(T_w^D)=w]\big] \leq \gamma
  \end{equation*}
\end{definition}

\begin{proposition}
  Tamaraw $\gamma$-privacy is strictly weaker than the notion of $(\varepsilon, 0)$-differential privacy.
\end{proposition}
\noindent
To prove the above proposition, we need to prove the following two lemmas.

\begin{lemma-numbered}
  There exists a Tamaraw $\gamma$-private defense mechanism that fails to satisfy $(\varepsilon, 0)$-differential privacy for any given value of $\varepsilon$.
\end{lemma-numbered}
\begin{proof}
%  Assume a closed-world setup of $n$ webpages.
    Consider a web service with a dataset of $n$ web pages.
    We propose the following defense mechanism, $D$, with two parameters $\alpha$ and $\beta$:
    \begin{enumerate}
    \item For the webpage $w_i: i=j$, $D$ reshapes it to the constant-rate pattern, $O_c$, with probability $\beta$. Otherwise, with probability $1-\beta$, it reveals the original traffic pattern of the webpage, $T_{w_{i=j}}$.
    \item For any webpage $w_i: i \in \{1, 2, \dots, j-1, j+1, \dots, n\}$, $D$ reshapes it to the constant-rate pattern, $O_c$, with probability $\alpha$ such that $\alpha > {e^{\varepsilon}}\beta$. Otherwise, $D$ reveals the original pattern of $w_i$, $T_{w_{i\neq j}}$, with probability $\alpha$.
    \end{enumerate}
    The probability that any attack can correctly identify the label for webpage $w_j$ is upper-bounded by:

    \begin{align*}
      & \Pr[A(T^{D}_{w_{i=j}}) = w_j]
      \\
      & = \Pr[A(T^D_{w_{i=j}}) = w_j | T^D_{w_{i=j}}=T_{w_{i=j}}]\Pr[T^D_{w_{i=j}}=T_{w_{i=j}}] +
      \\
      &~~~~\Pr[A(T^D_{w_{i=j}}) = w_j | T^D_{w_{i=j}}=O_c]\Pr[T^D_{w_{i=j}}=O_c]
      \\
      & \leq  1.(1-\beta) + \frac{1}{n}\beta = p_c^j
    \end{align*}
    For $(1- \frac{n\gamma - 1}{n-1}) < \beta$ we have: $p_c^j \leq \gamma$.
    \\
    Similarly, the probability that any attack can correctly classify $w_{i\neq j}$ is upper-bounded by $p_c^i = 1-\alpha + \frac{\alpha}{n}$, and for $(1- \frac{n\gamma - 1}{n-1}) < \alpha$ we have: $p_c^j \leq \gamma$.
    Therefore, for all values of $\alpha$ and $\beta$ such that $(1- \frac{n\gamma -
    1}{n-1}) < \beta < \alpha$, the probability that any attack can successfully
    guess victim traffic stream in both cases is less than $\gamma$, and
    the defense is uniformly $\gamma$-private.
    \\
    When the output of the algorithm is a constant pattern, $O_{c}$, with the probability $\beta$ the original webpage is $j$, and with probability $\alpha$, it can be any other webpages. Thus, we have:
    \begin{equation*}
    \log(\frac{\Pr[T_{w_{i\neq j}}^{D}=O_{c}]}{\Pr[T_{w_{i=j}}^{D}=O_{c}]})
    = \log(\frac{\alpha}{\beta}) > \varepsilon
    \end{equation*}
  Therefore, it fails to guarantee $\varepsilon$-differential privacy.
\end{proof}


\begin{lemma-numbered}
  A $(\varepsilon, 0)$-differentially private shaping algorithm is Tamaraw $\gamma$-private for:
  \begin{equation*}
    \varepsilon \leq \log(n\gamma)
  \end{equation*}
\end{lemma-numbered}
\begin{proof}
  For a given trace, $w$, the random variable $T_{w}^{DP}$ represents packet
  trace of $w$ after  a differentially private shaping mechanism is applied.
  \\
  The classification attack on shaped traffic analysis the shaped traces so it can be considered as post-processing of the results of a
  differentially private shaping mechanism (i.e. defense), and is differentially
  private. Therefore,
  we have:
  \begin{align*}
    \frac{\Pr[A(T_{w_{i}}^{DP}) = w_i]}{\Pr[A(T_{w_{j}}^{DP}) = w_i]} \leq e^
    {\varepsilon}
    \\
    \rightarrow \Pr[A(T_{w_{i}}^{DP}) = w_i] \leq e^
    {\varepsilon} .\Pr[A(T_{w_{j}}^{DP}) = w_i]
  \end{align*}
  Intuitively, this implies that the likelihood of the attacker correctly classifying the trace with label $i$ compared to incorrectly classifying it with label $j$ is bounded by $e^{\varepsilon}$.
  The above inequality is correct for all $w_j: j\in \{1, 2, \dots, n\}$, therefore we can calculate the summation over $j$.
  Extending the above equation we have:
  \begin{align*}
    n\times \Pr[A(T_{w_{i}}^{DP}) = w_i] \leq e^{\varepsilon}\sum_{j=1}^{n} \Pr[A(T_{w_{j}}^{DP}) = w_i] \\
    = e^{\varepsilon} \operatorname{Pr}_{A}[w_i]
  \end{align*}
  where $\Pr_{A}[w_i]$ is the probability that attack $A$ outputs the label
  $w_i$.
  Therefore, for any given trace $w_i$, the probability that any attack $A$, classifies it correctly is bounded by:
  \begin{equation*}
    \Pr[A(T_{w_{i}}^{DP}) = w_i] \leq \frac{e^{\varepsilon} \Pr_{A}[w_i]}{n}
  \end{equation*}
  Therefore, the probability that an attacker can guess the victim’s trace is bounded by:
  \begin{align*}
    \max_{w_i}{\Pr[A(T_{w_{i}}^{DP}) = w_i]} \leq \frac{e^{\varepsilon}}{n} \max_{w_i}{\operatorname{Pr}_{A}[w_i]} \leq \frac{e^{\varepsilon}}{n} \leq \gamma
  \end{align*}
\end{proof}
\noindent
Putting the two lemmas together, we prove that the notion of differential
privacy is strictly stronger than Tamaraw's.

Tamaraw's shaping method is almost the same as BuFLO~\cite{dyer2012peek}. The only difference it that instead of padding the total transmission time a threshold $T$, Tamaraw pads the total number of packets to multiples of a padding parameter $L$.
In other words, if the total packet number for a webpage $W$ is $n_W$, Tamaraw pads it to a size of $\lceil \frac{n_W}{L} \rceil \times L$.
As the value of $L$ increases, Tamaraw offers enhanced privacy; however, this also results in higher overheads.




\subsection{Adversarial Traffic Patterns}

\subsection{Traffic Shaping Frameworks}
\subsubsection{QCSD}

