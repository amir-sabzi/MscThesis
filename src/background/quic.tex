
QUIC is a modern transport layer protocol specifically developed to enhance the performance of HTTPS traffic~\cite{langley2017quic}.
QUIC provides transport functionalities of flow control, loss recovery, encryption, and congestion control, on the top of UDP.
QUIC is now standardized under RFC 9000~\cite{rfc9000}, and the details of design and implementation of the protocol is well beyond the scope of this thesis.
Each QUIC connection carries multiple QUIC streams.
QUIC's streams offer a lightweight and ordered way for applications to send byte-streams.
These streams can be either unidirectional or bidirectional, depending on the specific needs of the application.
Unlike TCP, QUIC alleviates head-of-the-line blocking delays, enabling a QUIC connection to efficiently handle flows from various applications as separate streams within a single connection. 
We specifically leverage this property of QUIC in design of our traffic shaping tunnel in \Cref{sec:tunnel-design}.
Within a connection, each stream is identified with a new ID.
Each stream has a unique header with 5 major fields: Type, Stream ID, Offset, and Data Length.
The stream header and its payload within a packet are encrypted, ensuring that adversaries are unable to discover the active streams in a connection.

% \subsection{QUIC Congestion and Flow Control}
% \subsubsection{Flow Control}
% To prevent a fast or malicious sender from overwhelming receivers' buffer, receivers must proactively limit the amount of data a sender can send.
% QUIC can perform flow control at two levels: The receiver controls the maximum amount of data the sender can send on a per-stream basis, as well as across all streams within a connection. 
% The receiver sets the initial limits for stream and connection flow controls during the handshake. 
% During the handshake, the receiver establishes the initial limits for both stream and connection flow controls. Throughout the connection, if the receiver needs to adjust the limits for a specific stream, it sends \texttt{MAX\_STREAM\_DATA} frames to the sender.
% Similarly, to change the limit for overall connection, the receiver sends \texttt{MAX\_DATA} frames to the sender.

% \subsubsection{Congestion Control}
% QUIC does not standardize any specific method of congestion control.
% Instead, it provides the necessary feedback mechanism to implement congestion control.
% The modular design of congestion control makes QUIC more flexible, allowing applications to implement a congestion mechanism that aligns with their specific requirements.
% As the default congestion control mechanism, however, QUIC uses a sender specified congestion controller similar to TCP NewReno~\cite{rfc6582}.
% Given the limitations of this thesis, a comprehensive explanation of TCP NewReno is outside its scope.
% For a more in-depth understanding of this mechanism, we kindly refer readers to the RFC, which provides detailed information and insights.
