We assume that the communicating parties are inside separate trusted private networks (\eg each node is behind a VPN gateway node), which an adversary cannot compromise.
We assume that all the application endpoints are non-malicious and do not leak the secrets themselves.
The adversary with control over the underlying network links (e.g., Internet Service Providers) can monitor, manipulate, and record the victim's traffic pattern.
In particular, it can precisely record the traffic shape---the sizes, timing, and direction of packets---on all links in the victim's traffic communication graph.
Furthermore, the adversary can drop, replay, or inject packets into the victim's traffic.
However, it cannot break standard cryptography, cannot compromise the victim's VPN, or cannot impersonate its clients or servers to directly interact with the victim (thus, no covert attacks).
We do not consider threats due to observing the IP addresses of packets.
We also do not consider the threat where a victim accidentally installs a malicious script in the browser, thus enabling an adversary to colocate with the
victim's application and observe its traffic~\cite{schuster2017beautyburst,mehta2022pacer}.
This is a reasonable assumption, since a colocated adversary can exploit many other direct or indirect channels for data leaks that will be far more efficient than {\nsc}s~\cite{kocher2018spectre, yarom2014flushreload, liu2015llcpractical, irazoqui2015ssa, vila2017loophole}.

In this project, we focus on the design of {\sys}'s traffic shaping tunnel
within a middlebox that can be placed in front of the gateway router of an
organization. 
Furthermore, we simulate shaping mechanism of {\sys} and evaluate its privacy and overheads.

{\sys}'s trusted computing base (TCB) includes all components in the organization's private network and the middleboxes.
We assume that the middleboxes do not directly leak application secrets to the adversary.
We assume the cryptographic libraries are side-channel free~\cite{almeida2016verifying}.

{\sys} does not address leaks of one application's sensitive data through the traffic shape of colocated applications transmitting only non-sensitive traffic.
Such leaks may arise, for instance, due to microarchitectural interference among the applications colocated on a host or among their flows if they pass through shared links.
Mitigating such leaks would require physically isolating the
applications and their flows.
For instance, a service serving both privacy-sensitive and privacy-insensitive clients may be partitioned into two instances, with each instance dedicated to serving clients with similar privacy requirements.
For the privacy-sensitive clients, we assume the corresponding service instance agrees to serve the clients according to their privacy requirements. 
Alternatively, end hosts could implement mitigations for various side channels to support colocated applications \cite{mehta2022pacer, page05partitionedcache, shi2011limiting, kim2012stealthmem, varadarajan2014scheduler, braun2015robust}. 
To mitigate leaks among colocated flows at shared links, {\sys} assumes that the network paths of the sensitive and non-sensitive flows are completely partitioned.
In practice, this limitation can be removed by combining {\sys}’s traffic shaping with TDMA scheduling \cite{beams2021ifs, vattikonda2012tdma}.


Under these assumptions, {\sys} prevents leaks of application secrets through the sizes and the delays between the network packets transmitted in either direction between the application endpoints.

